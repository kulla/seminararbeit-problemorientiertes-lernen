\documentclass[12pt,a4paper]{scrartcl}

\usepackage[utf8]{inputenc}
\usepackage{amsmath}
\usepackage{ngerman}
\usepackage[onehalfspacing]{setspace}

\title{Problemorientierter Unterricht}
\author{Stephan Kulla}

\begin{document}
\maketitle

\section{Problemorientierter Unterricht}

\subsection{Wozu problemorientierter Unterricht?}

Die Digitalisierung wird weitreichende Folgen und Auswirkungen auf die zukünftige Arbeitswelt haben\cite{dengler2015}. Wenn man beispielsweise der Studie von Osborne und Frey folgt \cite{frey2017}, so könnte in den USA in den nächsten 20 Jahren jeder zweite Job durch die Arbeit von Computer ersetzt werden. \cite{dengler2015} schätzen in ihrem Bericht, dass $15\%$ der Arbeitsplätze in Deutschland sehr wahrscheinlich durch die Einführung von Programmen wegfallen werden.

Damit stellt sich die Frage an Lehrkräfte, wie sie ihre Schülerinnen und Schüler auf die neue Arbeitswelt vorbereiten können. So argumentiert Günter Dueck, dass das Bildungssystem neue Kompetenzen fördern muss: Soziale Kompetent, Kommunikationsfähigkeit, Problemlösefähigkeit und fachliche Kompetenzen sind einige der vielen Fähigkeiten (Wort?), die in der zukünftigen Arbeitswelt benötigt werden (Zitat).

Problemorientierter Unterricht ist ein forgeschlagener Weg, um diese Kompetenzen bei Schülerinnen und Schülern zu fördern\cite{silver2004}. Es ist „Unterricht im Geist des Problemlösens“, wie Reusser es beschreibt\cite{reusser2005}. Schülerinnen und Schülern wird ein komplexes Problem präsentiert, welches sie eigenständig (oft in Gruppenarbeit) lösen\cite{silver2004}. Die Probleme sind dabei lebensnah und authentisch \cite{kunter2013} oder didaktisch ansprechend aufbereitet\cite{reusser2005}.

Problemorientierter Unterricht verfolgt das Ziel, transfährfähiges Wissen aufzubauen und fachspezifische Denk- und Lernstrategien zu fördern\cite{reusser2005}. Durch Gruppenarbeit können außerdem die sozialen und kommunikativen Kompetenzen gestärkt werden\cite{seidel2014}. Dabei kann problemorientierter Unterricht sowohl eingesetzt werden, um bereits erworbenes Wissen anzuwenden, als auch um neues Wissen zu erwerben\cite{reusser2005}.

\subsection{Entdeckendes Lernen und Stationenarbeit}

Entdeckendes Lernen und Stationenarbeit sind zwei konkrete Methoden, wie Probleme von den Schülerinnen und Schülern im Unterricht gelöst werden können\cite{kunter2013}. Bei der Methode des entdeckenden Lernens wird nach Hameyer und Rößer \cite{hameyer2008} wird in der Konfrontationsphase zunächst ein Problem präsentiert. Dieses wird in der Entdeckungsphase von den Schülerinnen und Schüler bearbeitet, wobei die Lehrkraft unterstützend tätig wird. Zum Abschluss werden in der Präsentationsphase die Ergebnisse vor der Klasse präsentiert und diskutiert.

Die Stationenarbeit (auch „Arbeit im Lernzirkel“ oder „Lernen an Stationen“ genannt) ist eine weitere Unterrichtsmethode, um problembasierten Unterricht umzusetzen\cite{hegele2008}. Nach Hegele \cite{hegele2008} kann diese Form des Unterrichts in fünf Phasen vollzogen werden:

\begin{enumerate}
  \item \emph{Hinführung:} Durch Präsentation des Problems soll das Interesse der Schülerinnen und Schüler geweckt werden.
  \item \emph{Rundgang:} Die Lehrerin bzw. der Lehrer stellt alle Stationen vor und gibt Hinweise, die nicht direkt aus den Arbeitsaufträgen ersichtlich sind. Auch auf Gefahren bei der Benutzung von Geräten oder auf vereinbarte Verhaltensweise kann hingewiesen werden.
  \item \emph{Arbeit an den Stationen:} Die Schülerinnen und Schüler arbeiten eigenständig oder in Gruppen an den Stationen.
  \item \emph{Zwischen- oder Schlusskreis:} Abschließende Gespräche zwischen den Arbeitsphasen, um Lernergebnisse zur Kenntnis zu nehmen, das Bewusstsein zu stärken, Teil einer größeren Lerngruppe zu sein oder Lernschwierigkeiten anzusprechen.
  \item \emph{Präsentation der Ergebnisse:} Abschließende Präsentation der Ergebnisse (beispielsweise auf Elternabende, im Klassenzimmer, etc.)
\end{enumerate}

\subsection{Warum problemorientierter Unterricht?}

Problemorientierter Unterricht passt zur konstruktivistischen Sichtweise auf Lernen, nach der Lernergebnisse aktiv durch den Lernenden selbst kontstruiert werden\cite{reiss2012}. Die 

\begin{itemize}
  \item Verarbeitungstiefe erhöht die Speicherung der neuen Informationen im Langzeitgedächtnis\cite{seidel2014}
  \item Erhöhte Autonimie der Schülerinnen und Schüler sollte Motivation erhöhen nach ...
  \item Förderung von Lernstrategien -> Föderung von selbstregulart
  \item Folgt der konstruktivistischen Tradition (TODO: cite) -> Theorie kurz vorstellen
  \item Gestützt vom Angebots-Nutzungs-Modell, dass erst die Lernaktivität an sich zu Unterrichts
  \item Begründung: situiertes Lernen \cite{reusser2005}

    \begin{itemize}
      \item Schulische Lernsituationen selten authentisch
    \end{itemize}

  \item \textbf{Anwendungsbeispiele}: TED-Vortrag des Biologie-Lehrers
  \item Flipped-Classroom

  \item \textbf{Zusammenhang mit fachdidaktischen Überlegungen:} Mathematik: Genetische Methode
  \item Physikunterricht: Nicht wegzudenken (Der andere Text)
\end{itemize}

\subsection{Forschungsergebnisse}

\begin{itemize}
  \item Problemorientierter Unterricht wirkt sich positiv auf die MINT-Leistung der Schülerinnen und Schüler aus\cite{seidel2016}
  \item Große Hoffnung in diese Methode\cite{kunter2013}
  \item Unterschiedliche Ergebnisse: Studien können Lernergebnisse nicht nachweisen\cite{kunter2013}
  \item Große Freiheiten bei wenig Vorwissen eher schädlich\cite{kunter2013}
  \item Es kann schnell zu einer Überforderung kommen\cite{kunter2013} -> Cognitive Load Theory / begrenztes Arbeitsgedächtnis
  \item Wenig Rückmeldungen sind auch schlecht\cite{kunter2013}
  \item Begrenztheit der Lernenden muss beachtet werden (in motivationaler, und kognitiv) // Fachfremde Tätigkeiten, die ablenken\cite{kunter2013}
  \item mittlere Effekt bei forschungszentriertem Lernen im Vergleich zu lehrerzentriertem Lernen
  \item Hoch bei epistemisch orientiertem Lernen d=0,75
  \item prozeduale und soziale Aktivitäten
  \item stärker lehrergesteuert besser als stärker schülergesteuert
  \item Unterstützung der Lehrkraft -> bessere Effekte
\end{itemize}

\section{Planung einer Seminarstunde zum problemorientierten Unterricht}

\subsection{Ziel der Seminarstunde}

Unser Ziel in der Seminarstunde war es, eine Einführung in das Thema des problemorientierenten Unterrichts zu geben. Die Studierenden sollen wiedergeben können, was problemorientierter Unterricht ist und wie durch Stationenarbeit problemorientierter Unterricht umgesetzt werden kann.

Uns ist besonders wichtig, dass die Studierenden folgende Forschungsergebnisse zu problemorientierten Unterricht kennen:

\begin{itemize}
  \item Entdeckendes Lernen ist effektiver als 
\end{itemize}

\section{Reflexion der Seminarstunde}

\section{Persönliches Fazit: Ausblick und Diskussion}

\section{TODOS}

\begin{itemize}
  \item Übersichrift Literatur
  \item Überschriften im Titel groß anzeigen
\end{itemize}

\bibliographystyle{plain}
\bibliography{literatur}

\end{document}
